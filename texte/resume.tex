%\resizebox{4cm}{!}{\Huge Résumé}
{\fontsize{30}{100}\selectfont Résumé}

\rule{15cm}{0.1em}

\vspace{1cm}

\thispagestyle{plain}

La tomographie par émission de positons (TEP) est une méthode d’imagerie clinique en forte expansion dans le domaine de l’oncologie. De nombreuses études cliniques montrent que la TEP permet, d’une part de diagnostiquer et caractériser les lésions cancéreuses à des stades plus précoces que l’imagerie anatomique conventionnelle, et d’autre part d’évaluer plus rapidement la réponse au traitement. Le raccourcissement du cycle comprenant le diagnostic, la thérapie, le suivi et la réorientation thérapeutique contribue à augmenter le pronostic vital du patient et maîtriser les coûts de santé.

La durée d’un examen TEP, de l’ordre de 5 à 10 minutes pour un champ de vue axial de l’ordre de 15 cm, ne permet pas de réaliser une acquisition sous apnée. La qualité des images TEP est par conséquent affectée par les mouvements respiratoires du patient qui induisent un flou dans les images. Les effets du mouvement respiratoire sont particulièrement marqués au niveau du thorax et de l’abdomen.

Plusieurs types de méthodes ont été proposés pour corriger les données de ce phénomène, mais elles demeurent lourdes à mettre en place en routine clinique. La problématique de la correction du mouvement respiratoire et le choix de la méthode appropriée sont des sujets d’actualité au sein de la communauté de médecine nucléaire. Des travaux récemment publiés proposent une évaluation de ces méthodes basée sur des critères de qualité tels que le rapport signal sur bruit ou le biais. Aucune étude à ce jour n’a évalué l’impact de ces corrections sur la qualité du diagnostic clinique. Ce problème pose des questions d’orientation stratégique et financière importantes.

Nous nous sommes focalisés sur la problématique de la détection des lésions du thorax et de l'abdomen de petit diamètre et faible contraste, qui sont les plus susceptibles de  bénéficier de la correction du mouvement respiratoire en routine clinique.

Nos travaux ont consisté dans un premier temps à construire une base d’images TEP qui modélisent un mouvement respiratoire non-uniforme, une variabilité interindividuelle et contiennent un échantillonnage de lésions de taille et de contraste variable. Ce cahier des charges nous a orientés vers les méthodes de simulation Monte Carlo qui permettent de contrôler l’ensemble des paramètres influençant la formation et la qualité de l’image. Une base de 15 modèles de patient a été créée en adaptant le modèle anthropomorphique XCAT sur des images tomodensitométriques (TDM) de patients. Cette base contient 280 lésions sphériques de 8 et 12 mm de diamètre réparties dans les poumons et le foie et dont le contraste a été calibré pour échantillonner la gamme de détection. Le signal respiratoire simulé est constitué d’un motif de 4 cycles mesurés sur des patients. Nous avons ensuite développé un protocole de simulation à l’aide du logiciel PET-SORTEO afin de générer les examens virtuels de chacun des modèles en considérant deux types d’acquisition : une acquisition dynamique en mode-liste synchronisée sur le signal respiratoire et une acquisition statique de même durée que l’acquisition dynamique mais correspondant au cas idéal d’un patient qui pourrait retenir sa respiration.

Nous avons en parallèle développé une stratégie originale d’évaluation des performances de détection. Cette méthode comprend un système de détection des lésions automatisé basé sur l'utilisation de machines à vecteurs de support (SVM) utilisant des caractéristiques fréquentielles. Les performances sont mesurées par l’analyse psychophysique des courbes free-receiver operating characteristics (FROC) que nous avons adaptée aux spécificités de l’imagerie TEP. L’évaluation des performances est réalisée sur deux méthodes prometteuses de correction du mouvement respiratoire, en les comparants avec les performances obtenues sur les images non corrigées ainsi que sur les images idéales sans mouvement respiratoire. 
Les résultats obtenus sont prometteurs et montrent une réelle amélioration de la détection des lésions après correction, qui approche les performances obtenues sur les images statiques. 

\textsc{Mots-Clefs :} TEP, oncologie, mouvement respiratoire, reconnaissance de formes, Computer aided detection (CAD), Séparateurs à vaste marge (SVM), Simulation Monte-Carlo