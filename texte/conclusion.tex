

{\fontsize{30}{100}\selectfont Conclusion}

%\resizebox{4cm}{!}{\Huge Conclusion}

\rule{15cm}{0.1em}

\vspace{1cm}

\thispagestyle{plain}

L'étude que nous avons réalisés a montré une amélioration globale de la
détectabilité des lésions, apportée par les techniques de correction du
mouvement respiratoire. Cependant, les résultats sont parfois contrastés
selon les deux techniques de correction du mouvement respiratoire, notamment
pour le poumon, ce qui semble indiquer que les organes réagissent différemment à
ces corrections.

L'approche région que nous avons utilisés lors de notre estimation est
originale, dans le sens où nous avons appliqué un estimateur région qui se
rapproche du modèle de détection humain. Cependant, il reste de nombreuses
possibilités d'amélioration, notamment au niveau de la complexité du modèle
utilisé et de la sélection des paramètres. 

De plus, la référence reste toujours l'observateur humain, et les résultats que
nous avons observés doivent être comparés avec les performances d'un observateur
humain, et validé sur des données cliniques. Nous avons montré qu'il était
possible de répondre à la problématique uniquement à partir de données simulées
et d'observateurs informatiques.