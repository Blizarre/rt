

\resizebox{10cm}{!}{\Huge Conclusion}

\rule{15cm}{0.1em}

\vspace{3cm}
%\section{Conclusion}

L'étude que nous avons réalisé a montré une amélioration globale de la détectabilité des lésions, apportée par les techniques de correction d'images. Cependant, les résultats sont parfois contrastés selon les techniques de correction du mouvement respiratoire, notamment pour le poumon, ce qui semble indiquer que les organes réagissent différemment à ces corrections.

L'approche région que nous avons utilisé lors de notre estimation est originale, dans le sens où nous avons appliqué un estimateur région qui se rapproche du principe de détection humain. Cependant, il reste de nombreuses possibilités d'amélioration, notamment au niveau de la complexité du modèle utilisé et de la sélection des paramètres. 

Cependant, la référence sera toujours l'observateur humain, et les résultats que nous avons observés doivent être comparés avec les performances d'un observateur humain, et validé sur des données cliniques. Nous avons montré qu'il était possible de répondre à la problématique uniquement à partir de données simulées et d'observateurs informatiques.