\section{FDM}

Faire la distribution des scores de FP et des VP pour comparer les méthodes et avoir quelquechose de plus eprtinent que la FDM
De plus, utilise le MAX !

ET-IM et ET-LOR ont des performances plus faibles à chaque fois. Cela pourrait s'expliquer par le fait que les lissages vont réduire la réponse fréquentielle et ainsi réduire la réponse du classifieur ?

\section{base d'apprentissage}

Idéalement, il faudrait utiliser une métrique type Free-ROC pour estimaer les perfs de chaque jeu de param (=> sur-apprentissage !!)
De plus, si base App != FROC : base app = perfs au 0

\section{Base de donnée}

Voir la thèse de simon studt, où il démonte l'approche modèle pour en extraire les limitations :

- Trop d'homogénéité

Mais pleins d'avantages (respirant)

\subsection{Paramètres}

Une étude de ce type demande  la fixation d'un nombre très important de paramètres. 
Dire que les paramètres peuvent encore être mieux sélectionnés maintenant que la chaine complète est réalisée.

Partie champ de mouvement (ET-LOR pas terrible) : pas spécialiste, a améliorer.


\section{SimuTDM}

Actuellement toutes les acquisitions sont réalisées en TEP/TDM. Avons tenté de générer base TEP/TDM, mais sans succès. Problème de qualité de simulation. Modèle trop pauvre, => images trop parfaites => plus besoin de la TEP

Grosse perte de temps, avec évaluation et prise en main simulateur TDM, embauche d'un stagiaire, etc.
mais cependant travail dans projet européen VIP, avec l'intégration de SINDBAD ainsi que de SORTEO. VIP va permettre de générer des bases de données plus réalistes.
Approche modèle (VIP) bien, mais demande modèles + complexes (thésard avec textures)

Détection => cas de tumeurs sphériques dans notre cas, mais pas une limitation, seulement un choix. Citer Travaux d'amandine sur la simulation => puis onco\_PET



AAAAAAAAAAATTTTTTTTTENNNNTION

Approche cluster -> originale, pas satisfait des ROC ou juste sensib/specif, => FROC, même si pas idéal. Avoir une réflexion sur les matriques, et les paramètres.


A valider par des observateurs humains