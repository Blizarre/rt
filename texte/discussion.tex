\chapter{Discussion et Conclusion}

Les contributions de cette thèse sont doubles. Elles portent, à la fois sur la
création d'une base de données de patients simulés respirant, en TEP dynamique, et
sur un ensemble d'outils d'évaluation des performances de détection
des lésions.  
 
Nous allons tout d'abord présenter une synthèse des résultats, puis les limites et perspectives de notre système de détection ainsi que de la base de données.

\section{Synthèse des résultats}

Les résultats que nous avons présentés montrent une différence importante de
détectabilité des lésions dans les images non corrigées et dans les images statiques
avec tous les critères utilisés. Ce résultat conforte la fiabilité du système de
détection que nous avons développé. Les détections réalisées dans les images
corrigées montrent des performances de détectabilité intermédiaires entre celles obtenues dans les images statiques et celles avec obtenues dans les images dynamiques non corrigées. 

Pour le poumon, les courbes Free-ROC montrent que la détectabilité des
lésions est bien retrouvée pour la correction du mouvement respiratoire TE-IM, avec une sensibilité de 62\% contre 63\% sur les images statiques, pour
20 faux positifs en moyenne par image. La méthode de correction du mouvement TE-MS apporte une sensibilité de seulement 50\% dans les mêmes conditions. En revanche, les FDM
issues de l'analyse JAFROC montrent une amélioration semblable pour
les deux techniques de correction du mouvement. Pour le foie, les courbes F-ROC
montrent que les techniques de correction du mouvement respiratoire ont des performances intermédiaires par rapport à celles des images statiques et non
corrigées avec, par exemple, pour 9 faux positifs par image une sensibilité de
58\% pour les images statiques, 53\% pour la méthode TE-IM, 48\% pour la méthode TE-MS et 37\% pour
les images non corrigées. Cependant, les FDM issues des images corrigées du foie
ne montrent pas d'améliorations notables de la détectabilité apportée par la
correction du mouvement par rapport à celle des images non corrigées. 

Les performances observées à l'aide des courbes Free-ROC sont très différentes
 entre les deux organes, notamment le nombre de faux positifs moyen qui est
beaucoup plus faible pour le foie que pour le poumon. Cette différence
s'explique par les différences de volumes des
organes. En effet, le volume des poumons est approximativement deux fois plus
important que celui du foie dans le modèle XCAT que nous avons utilisé.

Les courbes Free-ROC sont plus pertinentes que les FDM pour la
détection, mais il n'existe pas de métrique permettant de synthétiser l'ensemble des performances, comme peut le faire l'aire sous la courbe pour les étude ROC. Il est intéressant de constater que les deux métriques que nous
avons utilisées (F-ROC et FDM JAFROC) montrent des résultats parfois contradictoires. Cela peut
s'expliquer par la différence de l'information qu'elles mesurent. L'analyse JAFROC
ne prend en compte que le faux positif de plus haut score pour chaque image,
tandis les courbes F-ROC utilisent le nombre de faux positifs dépassant un
certain seuil. Les FDM sont naturellement plus difficiles à interpréter et sont,
par nature, plus sensibles au bruit car elles utilisent des maxima de distributions, et sont
donc très dépendantes des valeurs extrêmes. Cependant, cette métrique renseigne
sur la différence de détectabilité entre les vrais positifs et le
faux positifs de plus haut score. C’est une métrique qui est handicapée par la faible quantité d’exemples à disposition, comme
le montrent les barres d’erreur de la figure~\ref{fig:fom_mod}.



Les résultats que nous avons obtenus montrent néanmoins, sans ambiguïté, que
la détectabilité des lésions est améliorée par la correction du mouvement
respiratoire, notamment en utilisant la correction TE-IM. La détectabilité des
lésions indiquée par la courbe F-ROC est systématiquement supérieure sur les
images corrigées par rapport à celle des images non corrigées. Cependant la correction TE-MS a des performances inférieures à celles observées
pour TE-IM, ce qui est en contradiction avec les résultats présentés dans la
littérature sur la quantification des
lésions~\cite{lamare2007list,qiao2006motion}. Il faut noter que nous n'avons pas
réalisé notre
étude sur les mêmes principes, en privilégiant une approche détection sur la
quantification. Ces deux problématiques sont différentes et demandent des
paramétrages spécifiques. De plus, nous n'avons pas utilisé la même
implémentation de l'estimateur de mouvement et du correcteur d'image que
les auteurs, ce qui pourrait expliquer les différences observées. La précision de
l'estimation de mouvement est aussi différente, car nous utilisons un
espacement entre les noeuds de la grille B-spline de 17 cm en X et en Y contre 2.5 cm pour
l'évaluation de~\cite{lamare2007list}.


% \section{FDM}
% 
% Faire la distribution des scores de FP et des VP pour comparer les méthodes et avoir quelque chose de plus pertinent que la FDM
% De plus, utilise le MAX !
% 
% Nous avons constaté que la mesure de détectabilité à l'aide de la FDM sur les images corrigées techniques de 
% 
% correction du mouvement respiratoire 
% TE-IM et TE-LOR ont des performances plus faibles à chaque fois. Cela pourrait s'expliquer par le fait que les lissages vont réduire la réponse fréquentielle et ainsi réduire la réponse du classifieur ?

\section{Limites et perspectives}

%Nous avons présenté notre méthodologie pour la création de la base d'apprentissage. Nous avons retenu le jeu de paramètres qui maximise la sensibilité, mais comme nous l'avons signalé dans la partie correspondante, nous utilisons les mêmes données pour estimer les paramètres de la base d'apprentissage et pour réaliser la mesure de performances finales. Cela crée un biais positif, et il serait intéressant de disposer de plus de données pour pouvoir réduire ce biais ou le mesurer.

\subsection{Système de détection}

La métrique utilisée pour choisir les paramètres du système CAD
(voir~\ref{lab:paramClassif}) est très différente de celle utilisée pour
mesurer les performances de détection sur les différents jeux d'images de la base
(voir~\ref{lab:selectionSeuil}). L'étude de l'optimisation des paramètres du
système CAD est basée sur une validation croisée qui ne considère que le
voxel central des lésions. Si le score de ce voxel est supérieur à 0,
alors la lésion est considérée comme correctement détectée. Les
performances de détectabilité sont, quant à elles, basées sur une approche
région, en travaillant sur des agrégats. De plus, les métriques FDM et
Free-ROC utilisent des seuils variables pour déterminer les performances de chaque
technique. 

Il serait intéressant de travailler sur une métrique ``basée région'' pour la
recherche des meilleurs paramètres de la base d'apprentissage mais, dans ce cas,
l'évaluation des performances pour chaque jeu de paramètres nécessiterait de traiter
toutes les images, ce qui représente plusieurs heures de calculs par jeu de
paramètres. La recherche exhaustive que nous avons réalisée demande l'évaluation
de 900 triplets $(c, \gamma, j)$. ll n'est donc pas possible d'utiliser la même
métrique, à moins d'opérer une étude séparée sur les algorithmes d'optimisation à utiliser, et de
réaliser les calculs sur une grille. 

Il serait également intéressant d'augmenter la taille de la base de données
pour pouvoir éviter le phénomène de sur-apprentissage (optimisation
des paramètres pour le jeu de données et non pour les classes), qui est
relativement important dans notre étude, où les données d'apprentissage et
de tests sont extraites de la même base de données. Cela engendre un biais
positif sur les résultats. Ce biais n'est cependant pas très grave dans notre cas car
il affecte toutes les modalités, et nous ne sommes pas intéressés par les
performances absolues, mais par les performances relatives.
Pour éliminer ce biais, il faudrait entre autre, que la base de données utilisée
pour le réglage des paramètres soit différente de celle utilisée pour réaliser
l'estimation des performances.

Les performances que nous avons observées, avec des taux de détection de lésions
de l'ordre de 60\%, sont cohérentes avec la calibration réalisée sur les
activités des lésions. Le nombre de faux positifs par image que nous observons
pour ces niveaux de performances (entre 8 et 15) est trop important pour un
système d'assistance à la détection clinique, mais ce n'était pas notre
objectif.
 
%Idéalement, il faudrait utiliser une métrique type Free-ROC pour estimer les perfs de chaque jeu de param (=> sur-apprentissage !!)
%De plus, si base App != FROC : base app = perfs au 0

\subsection{Base de données}

Nous avons réalisé notre base de données à partir du modèle anthropomorphique
XCAT de Paul Segars, principalement pour sa capacité à s'adapter à des
morphologies différentes et sa prise en charge native d'un modèle respiratoire
complexe. 

Cependant, cette approche par modèle a des limitations connues, notamment
l'homogénéisation de l'activité des organes. En effet, contrairement à
l'imagerie TDM, où les coefficients d'atténuation sont relativement constants
dans les tissus et entre les patients, en TEP l'activité mesurée pour un même
organe varie entre les patients, et n'est pas constante à l'intérieur d'un même
organe. Les différences d'activités entre les patients s'expliquent par la
quantité de traceur injecté, d'éventuelles inflammations, le taux de glycémie
(pour le $^{18}FDG$), ou encore le temps écoulé depuis l'injection.  

Nous avions prévu, au début de la création de la base d'intégrer une variabilité
de captation inter-patients, mais nous avons dû y renoncer par manque de temps.
De plus, incorporer la variabilité de captation intra-organe pour les patients
n'aurait pas été possible car le modèle utilisé est trop simpliste.  

Dans la base de données que nous avons simulée, nous nous sommes restreints à
utiliser des lésions sphériques de petit diamètre (8 et 12 mm). Aux résolutions
permises par la TEP, cette approximation est suffisante pour modéliser
correctement les fixations observées en clinique. Cependant cette limite est
uniquement liée à notre choix de sujets, et n'est pas représentative des
possibilités du simulateur. Nous avons notamment travaillé sur des lésions de
formes différentes dans le cadre de la publication
de~\cite{le2009incorporating}.  


Du point de vue des choix de paramètres réalisés, une étude de
ce type demande la fixation d'un nombre très important de
paramètres, qui ont tous un impact sur les résultats. Nous avons justifié le
choix d'un maximum d'entre eux, mais ils sont souvent interdépendants, ce qui
engendre des problèmes impossibles à optimiser de manière exhaustive.  

Une évolution de nos travaux pourrait être de mettre en place une réflexion et
une refonte du système d'optimisation des paramètres en se basant sur la théorie
des plans d'expériences. Cela demanderait cependant des travaux supplémentaires
pour évaluer de manière précise les interactions entre chaque type de paramètres
et son implication dans le résultat final. 


\section{Imagerie TEP/TDM}

En routine clinique, les acquisitions sont maintenant réalisées conjointement
avec la TDM. Le rapprochement des scanners TEP et TDM sur le même appareil a
permis d'éviter la réalisation de la carte de transmission et fournit au
praticien des informations anatomiques qui permettent d'améliorer la précision
du diagnostic. Nous avions prévu au début de la thèse d'inclure la simulation
d'images TDM, et d'utiliser conjointement les deux modalités pour l'évaluation
de la détectabilité des lésions.  

Un partenariat avait été mis en place avec le laboratoire Leti du CEA de
Grenoble portant sur l'utilisation de leur simulateur SINDBAD. Nous avons passé
plusieurs mois à l'évaluer et à adapter nos données pour le valider. En effet, ce simulateur
 a été développé à l'origine pour le contrôle non destructif, et n'avait
jamais été utilisé pour simuler des données cliniques où les géométries, ainsi
que les énergies utilisées, sont totalement différentes. Nous avions embauché un
stagiaire ingénieur pour valider les simulations par rapport à des mesures
réelles sur des fantômes~\cite{leduvehat}. Cependant, les images que nous sommes parvenues à simuler étaient trop
``parfaites'' pour pouvoir être utilisées conjointement avec les images TEP. En
effet, le modèle que nous utilisons (XCAT) est trop peu détaillé pour une
modalité comme la TDM, qui dispose d'une résolution inférieure au millimètre. Les
inclusions étaient beaucoup trop visibles dans les tissus, en particulier
au niveau pulmonaire, car ces derniers n'étaient pas assez complexes pour
pouvoir les dissimuler malgré la modélisation des bronchioles et du système
sanguin. En effet, les images TDM réelles montrent fréquemment des artefacts
physiologiques qui doivent être éliminés par le radiologue, ou encore,
notamment pour le poumon, des textures très complexes que nous ne sommes pas
parvenus à reproduire. Malgré notre travail avec l'auteur du modèle, il n'a pas
été possible d'améliorer suffisamment la complexité du modèle pour obtenir des
résultats satisfaisants. 

Malgré ces difficultés, et en prenant la suite de nos travaux, le simulateur est
actuellement en cours d'intégration dans le projet  VIP ( Virtual Imaging Platform) porté par le laboratoire CREATIS (\url{http://www.creatis.insa-lyon.fr/vip}), où il devrait être
utilisé pour la simulation médicale. Le but de ce projet est de réaliser
une interface commune pour plusieurs simulateurs et les modèles associés. Une
interface en cours de développement permettra de réaliser les simulations sur
une grille de calculs (actuellement la grille Égée). Les simulateurs que nous
avons utilisés (PET-SORTEO et SINDBAD) sont en cours d'intégration ainsi
que le modèle XCAT. Des travaux sont en cours pour améliorer le réalisme des
modèles, notamment grâce à l'ajout d'informations de texture.  

\section{Conclusion}

L'étude que nous avons réalisée a montré une amélioration globale de la
détectabilité des lésions, apportée par les techniques de correction du
mouvement respiratoire. Cependant, les résultats sont parfois contrastés
selon les deux techniques que nous avons évaluées, notamment
pour le poumon, ce qui semble indiquer que les organes réagissent différemment à
ces corrections.

L'approche région que nous avons proposée et intégrée à notre système d'évaluation est
originale, dans le sens où nous avons appliqué un estimateur région qui se
rapproche du modèle de détection humain. Cependant, il reste de nombreuses
possibilités d'amélioration, notamment au niveau de la complexité du modèle
utilisé et de la sélection des paramètres. 

De plus, la référence reste toujours l'observateur humain, et les résultats que
nous avons observés doivent être comparés avec les performances d'un observateur
humain, et validés sur des données cliniques. Nous avons montré qu'il était
possible de répondre à la problématique uniquement à partir de données simulées
et d'observateurs informatiques.

%Actuellement toutes les acquisitions sont réalisées en TEP/TDM. Avons tenté de générer base TEP/TDM, mais sans succès. Problème de qualité de simulation. Modèle trop pauvre, => images trop parfaites => plus besoin de la TEP

%Grosse perte de temps, avec évaluation et prise en main simulateur TDM, embauche d'un stagiaire, etc.

%Cette étape nous a fait perdre beaucoup de temps, mais a permis l'inclusion du simulateur dans le projet ViP, où le travail porte actuellement 

%mais cependant travail dans projet européen VIP, avec l'intégration de SINDBAD ainsi que de SORTEO. VIP va permettre de générer des bases de données plus réalistes.
%Approche modèle (VIP) bien, mais demande modèles + complexes (thésard avec textures)


%Détection => cas de tumeurs sphériques dans notre cas, mais pas une limitation, seulement un choix. Citer Travaux d'amandine sur la simulation => puis onco\_PET



%AAAAAAAAAAATTTTTTTTTENNNNTION

%Approche cluster -> originale, pas satisfait des ROC ou juste sensib/specif, => FROC, même si pas idéal. Avoir une réflexion sur les métriques, et les paramètres.


%A valider par des observateurs humains


