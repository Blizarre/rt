%\resizebox{4cm}{!}{\Huge Introduction}

{\fontsize{30}{100}\selectfont Introduction}

\rule{15cm}{0.1em}

\vspace{1cm}
\thispagestyle{plain}

La tomographie par émission de positons (TEP) couplée à un tomodensitomètre (TDM) est une méthode d’imagerie clinique en forte expansion dans le domaine de l’oncologie. De nombreuses études cliniques montrent que la TEP permet, d’une part de diagnostiquer et caractériser les lésions cancéreuses à des stades plus précoces que l’imagerie anatomique conventionnelle, et d’autre part d’évaluer plus rapidement la réponse au traitement. Le raccourcissement du cycle comprenant le diagnostic, la thérapie, le suivi et la réorientation thérapeutiques contribue à augmenter le pronostic vital du patient et maîtriser les coûts de santé.

L’examen TEP/TDM au FDG apparaît désormais indispensable pour évaluer la réponse thérapeutique des patients atteints de lymphome. Les résultats préliminaires sont également très encourageants dans les tumeurs solides, en particulier les cancers pulmonaires, digestifs et de la sphère ORL~\cite{cachin2006evaluation}. La TEP pourrait prédire très précocement la réponse et ainsi permettre une réadaptation du schéma thérapeutique. Enfin de nouveaux traceurs TEP ouvrent la voie de l’imagerie moléculaire qui semble extrêmement prometteuse en thérapie génique et dans le suivi du traitement ciblé du cancer.

La durée d’un examen TEP, de l’ordre de 5 à 10 minutes pour un champ de vue axial de l’ordre de 15 cm, ne permet pas de réaliser une acquisition sous apnée. La qualité des images TEP est par conséquent affectée par les mouvements respiratoires du patient qui induisent un flou dans les images. Les effets du mouvement respiratoire sont particulièrement marqués au niveau du thorax et de l’abdomen. Deux types de méthode ont été proposés pour corriger les données de ce phénomène. Le premier type est basé sur l’utilisation d’acquisitions synchronisées sur la respiration produisant plusieurs images non affectées par la respiration et correspondant à différents instants du cycle respiratoire~\cite{nehmeh2002effect}\cite{boucher2004respiratory}. Cette méthodologie se base sur l’hypothèse  d’une grande régularité du signal respiratoire de synchronisation~\cite{boucher2004respiratory}, et conduit à des images d’une qualité statistique limitée car chacune ne contient qu'une fraction réduite du nombre de photons détectés pendant toute l’acquisition~\cite{visvikis2004evaluation}. Le deuxième type de méthode se base sur l’utilisation d’un scanner 4D TEP/TDM du patient permettant de modéliser les mouvements respiratoires. Ce modèle est alors intégré lors du processus de reconstruction tomographique~\cite{lamare2007list}. Ce type de méthode permet de pouvoir conserver la statistique de l’image mais nécessite une information anatomique dynamique assez lourde à acquérir en routine clinique.

La problématique de la correction du mouvement respiratoire et le choix de la méthode appropriée sont des sujets d’actualité au sein de la communauté de médecine nucléaire. Des travaux récemment publiés proposent une évaluation de ces méthodes basée sur des critères de qualité tels que le rapport signal sur bruit ou le biais~\cite{visvikis2004evaluation}. Aucune étude à ce jour n’a évalué l’impact de ces corrections sur la qualité du diagnostic clinique. Ce problème pose des questions d’orientation stratégique et financière importantes, puisque le second type de méthode requiert l’acquisition de données TEP/TDM dynamiques, très peu accessibles à l’heure actuelle.

C'est dans l'optique de résoudre ce problème que j'ai commencé ma thèse sur l'\textbf{Estimation des apports de la correction du mouvement respiratoire en oncologie TEP}.



La première partie de ca manuscrit va présenter l'imagerie TEP. Nous détaillerons les principes physiques de la TEP, le déroulement des acquisitions ainsi que les algorithmes de reconstruction utilisés pour former les images.

Ensuite, nous présenterons un état de l'art sur le mouvement respiratoire et sa correction dans les images TEP.

La troisième partie portera sur l'évaluation des performances de détections et sur les systèmes de détection utilisés pour mesurer ces performances.

Enfin, nous présenterons nos travaux : en premier lieu nous présenterons la base de données que nous avons créée pour réaliser les mesures de performances, puis la méthode utilisée pour répondre à la problématique, et enfin les résultats que nous avons obtenus.

La dernière partie corresponds aux discussions et perspectives sur notre travail.